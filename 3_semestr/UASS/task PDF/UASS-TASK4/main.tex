\documentclass[12pt]{article}
\usepackage{graphicx}
\usepackage{amsmath}
\usepackage{caption}
\usepackage{subcaption}
\usepackage{tikz}
\usepackage{float}

\title{Analýza dynamických sítí: email-dnc}
\author{Richard Ficek}
\date{\today}

\begin{document}

\maketitle

\section{Úvod}
Tato práce se zaměřuje na analýzu dynamických sítí na základě dvou různých datasetů: \texttt{email-dnc}. V této analýze budeme sledovat, jak se mění struktura sítě v čase, jak se vyvíjí průměrný stupeň, průměrný vážený stupeň, počet komunit, velikost komunit, a souvislé komponenty.

\section{Metodika}
Data byla rozdělena na 5 snímků, každý představující určitý časový okamžik. Pro každý snímek byla provedena analýza následujících metrik:
\begin{itemize}
    \item Průměrný stupeň sítě.
    \item Průměrný vážený stupeň sítě.
    \item Počet komunit a souvislých komponent.
    \item Identifikace vrcholů s vysokým stupněm a váženým stupněm.
    \item Vývoj vybraného vrcholu v čase.
    \item Vizualizace jednotlivých snímků.
\end{itemize}

\section{Výsledky}
\subsection{email-dnc}
\subsubsection{Průměrný stupeň sítě}
Pro dataset \texttt{email-dnc} jsme spočítali průměrný stupeň sítě pro každý z 5 snapshotů. Výsledek ukazuje, jak se mění propojenost mezi jednotlivými vrcholy v čase.

\begin{table}[htbp]
\centering
\resizebox{\textwidth}{!}{
\begin{tabular}{|c|c|c|c|c|c|c|c|}
\hline
\textbf{Snapshot} & \textbf{Nodes} & \textbf{Edges} & \textbf{Average Degree} & \textbf{Avg Weighted Degree} & \textbf{Communities} & \textbf{Average Community Size} & \textbf{Max Community Size} \\
\hline
1 & 834 & 1731 & 4.151 & 18.83 & 19 & 43.89 & 810 \\
2 & 10666 & 2408 & 4.518 & 44.20 & 26 & 41.00 & 1032 \\
3 & 1302 & 3133 & 4.813 & 72.37 & 32 & 40.69 & 1258 \\
4 & 1503 & 3725 & 4.957 & 104.48 & 35 & 42.94 & 1456 \\
5 & 1891 & 4465 & 4.722 & 124.57 & 41 & 46.12 & 1833 \\
\hline
\end{tabular}
}
\caption{Vývoj průměrného stupně sítě \texttt{email-dnc} v čase.}
\end{table}

\subsubsection{Vizualizace jednotlivých snímků}

Pro každý snímek jsme vytvořili vizualizace, které ukazují vývoj struktury sítě v čase. Tyto vizualizace zachycují dynamiku komunit a propojení vrcholů.

\begin{figure}[htbp]
\centering
\includegraphics[width=0.7\textwidth]{snap1.png}
\caption{Snímek 1: Vývoj struktury sítě.}
\end{figure}

\begin{figure}[htbp]
\centering
\includegraphics[width=0.7\textwidth]{snap2.png}
\caption{Snímek 2: Vývoj struktury sítě.}
\end{figure}

\begin{figure}[htbp]
\centering
\includegraphics[width=0.7\textwidth]{snap3.png}
\caption{Snímek 3: Vývoj struktury sítě.}
\end{figure}

\begin{figure}[htbp]
\centering
\includegraphics[width=0.7\textwidth]{snap4.png}
\caption{Snímek 4: Vývoj struktury sítě.}
\end{figure}

\begin{figure}[htbp]
\centering
\includegraphics[width=0.7\textwidth]{snap5.png}
\caption{Snímek 5: Vývoj struktury sítě.}
\end{figure}

\end{document}
