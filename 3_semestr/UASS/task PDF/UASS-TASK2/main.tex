\documentclass[a4paper,12pt]{article}
\usepackage[utf8]{inputenc}
\usepackage[czech]{babel}
\usepackage{graphicx}
\usepackage{amsmath}
\usepackage{hyperref}
\usepackage{geometry}
\usepackage{booktabs}
\usepackage{float} % Pro lepší umístění obrázků a tabulek
\usepackage{caption} % Upravené popisky
\geometry{margin=1in}

\title{\textbf{Analýza sociálních sítí v Gephi}\\ Úkol č. 2}
\author{Richard Ficek}
\date{\today}

\begin{document}

\maketitle

\tableofcontents
\newpage

\section{Vizualizace podle prezentací}

\subsection{Les Miserables}
\begin{figure}[H]
    \centering
    \includegraphics[width=0.9\textwidth]{lemis.png}
    \caption{Vizualizace sítě \textit{Les Miserables}.}
    \label{fig:power}
\end{figure}

\subsection{Dolphin social network}
\begin{figure}[H]
    \centering
    \includegraphics[width=0.9\textwidth]{dolphins.png}
    \caption{Vizualizace sítě \textit{Dolphin social network}.}
    \label{fig:power}
\end{figure}


\subsection{Book about US politc}
\begin{figure}[H]
    \centering
    \includegraphics[width=0.9\textwidth]{polbooks.png}
    \caption{Vizualizace sítě \textit{Book about US politc}.}
    \label{fig:power}
\end{figure}


\subsection{American College Football}
\begin{figure}[H]
    \centering
    \includegraphics[width=0.9\textwidth]{football.png}
    \caption{Vizualizace sítě \textit{American College Football}.}
    \label{fig:power}
\end{figure}


%-----------------------------------------------------------
\section{Opakování či odlišení}


V síťových modelech, jako jsou \textbf{Les Miserables}, \textbf{Dolphin social network}, \textbf{Books about US politics} a \textbf{American College Football}, se mohou opakovat různé vlastnosti a struktury. Například v \textbf{Les Miserables} existují vrcholy (postavy) spojené váženými hranami, jejichž váha odpovídá četnosti interakcí mezi postavami. V \textbf{Dolphin social network} se vrcholy (delfíni) spojují na základě četnosti společného pozorování, přičemž opakující se vztahy mezi delfíny indikují silné sociální vazby. V \textbf{Books about US politics} se opakují vzory nákupů knih, přičemž časté nákupy určitých knih společně jsou reprezentovány hranami mezi nimi. V \textbf{American College Football} se opakují vztahy mezi týmy, kdy jsou pravidelné zápasy mezi týmy v rámci konferencí, přičemž geografie a konferenční struktura ovlivňují četnost utkání. Tyto sítě se od sebe liší v typu vztahů mezi vrcholy (vážené vs. nevážené hrany), typu vrcholů (postavy, delfíni, knihy, týmy) a jejich reprezentaci (např. barvy vrcholů pro komunity, velikost vrcholů podle četnosti interakcí či zápasů).

\newpage
%-----------------------------------------------------------
\section{Popis vybrané sítě}
\noindent
Tato analýza se zaměřuje na neorientovanou, neváženou síť, která reprezentuje topologii energetické sítě západních Spojených států.

\bigskip
\noindent
\textbf{Popis komponent:}
\begin{itemize}
    \item \textbf{Vrcholy (Nodes)}: Elektrické uzly, jako jsou elektrárny, transformátory nebo rozvodny, které tvoří součást energetické infrastruktury.
    \item \textbf{Hrany (Edges)}: Elektrická vedení, která propojují jednotlivé uzly v síti a umožňují přenos elektrické energie mezi různými částmi systému.
\end{itemize}

\bigskip
\noindent
\textbf{Klíčové vlastnosti sítě:}
\begin{itemize}
    \item Síť zahrnuje jak hustě propojené oblasti, tak řídce propojené regiony, což odráží geografickou distribuci a fyzická omezení.
    \item Propojení mezi geograficky blízkými uzly je častější, zatímco vzdálenější uzly jsou spojeny méně často kvůli nákladům na budování přenosových vedení.
    \item Barvy vrcholů mohou označovat regionální skupiny nebo specifické části infrastruktury. Velikost vrcholů reflektuje důležitost uzlů (například počet připojených vedení).
\end{itemize}

%-----------------------------------------------------------
\section{Statistiky sítě}
\begin{table}[H]
    \centering
    \renewcommand{\arraystretch}{1.2} % Zvětší výšku řádků
    \begin{tabular}{l|r}
    \toprule
    \textbf{Metrika} & \textbf{Hodnota} \\
    \midrule
    Počet uzlů (Nodes) & 4941 \\
    Počet hran (Edges) & 6594 \\
    Průměrný stupeň (Average Degree) & 2.669 \\
    Modularity & 0.934 \\
    Počet komunit & 37 \\
    \bottomrule
    \end{tabular}
    \caption{Klíčové statistiky sítě \textit{Western States Power Grid}.}
    \label{tab:power_stats}
\end{table}

\bigskip
\noindent
Tabulka \ref{tab:power_stats} shrnuje základní statistiky sítě. Velká modularita (0.934) naznačuje přítomnost dobře definovaných komunit, což odpovídá logickému rozdělení energetické infrastruktury podle geografických oblastí.

%-----------------------------------------------------------
\section{Vizualizace sítě}
\subsection{Western States Power Grid of the United States}
\begin{figure}[H]
    \centering
    \includegraphics[width=0.9\textwidth]{ukol2.png}
    \caption{Vizualizace sítě \textit{Western States Power Grid of the United States}.}
    \label{fig:power}
\end{figure}


\end{document}
