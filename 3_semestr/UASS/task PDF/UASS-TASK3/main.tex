\documentclass[12pt]{article}
\usepackage{graphicx}
\usepackage{amsmath}
\usepackage{caption}
\usepackage{subcaption}
\usepackage{float}

\title{Analýza dynamických sítí: Email-Enron}
\author{Richard Ficek}
\date{\today}

\begin{document}

\maketitle

\section{Úvod}
Tato práce se zaměřuje na analýzu dynamických sítí na základě datasetu \texttt{Email-Enron}. Cílem je:
\begin{itemize}
    \item Zkonstruovat síť z dat.
    \item Prořídnout síť pomocí filtrace podle váhy hran.
    \item Identifikovat významné vrcholy (uživatele) v síti na základě metrik.
    \item Najít komunity v síti a analyzovat jejich velikost a strukturu.
\end{itemize}

\section{Metodika}
Postup analýzy probíhal v následujících krocích:
\begin{enumerate}
    \item \textbf{Načtení a zpracování dat:} Data obsahují dvojice vrcholů (uživatelů) a počty interakcí mezi nimi (váhy hran).
    \item \textbf{Konstrukce sítě:} Z dat byla vytvořena síť pomocí Python knihovny \texttt{networkx}. Vrcholy reprezentují uživatele a hrany interakce mezi nimi, přičemž váha hrany odpovídá počtu interakcí.
    \item \textbf{Prořídnutí sítě:} Síť byla prořídnuta odstraněním hran s nízkou váhou a ponecháním pouze těch, které nejvíce přispívají k propojenosti.
    \item \textbf{Identifikace významných vrcholů:} Pomocí metrik, jako je stupeň vrcholu (\textit{degree}) a meziessí (\textit{betweenness centrality}), byly identifikovány klíčové uzly.
    \item \textbf{Detekce komunit:} Louvainův algoritmus byl použit pro nalezení komunit v síti.
    \item \textbf{Vizualizace sítě:} Vyčištěná síť a komunity byly vizualizovány v Gephi.
\end{enumerate}

\section{Výsledky}
\subsection{Síťová analýza}
Po konstrukci sítě a prořídnutí byly spočítány klíčové statistiky. Tabulka~\ref{tab:network-stats} shrnuje základní charakteristiky původní i vyčištěné sítě:

\begin{table}[H]
\centering
\begin{tabular}{|c|c|c|}
\hline
\textbf{Parametr} & \textbf{Původní síť} & \textbf{Vyčištěná síť} \\
\hline
Počet uzlů & 36692 & 296 \\
Počet hran & 183831 & 500 \\
\hline
\end{tabular}
\caption{Statistiky sítě před a po prořídnutí.}
\label{tab:network-stats}
\end{table}


Vyčištěná síť byla uložena do souboru \textbf{cleaned_network gexf} pro další analýzu.


\subsection{Významní uživatelé}
Významné uzly byly identifikovány na základě stupně vrcholu (\textit{degree}) a meziessí (\textit{betweenness centrality}). Tabulka~\ref{tab:top-nodes} ukazuje 5 nejvýznamnějších uzlů podle těchto metrik:

\begin{table}[H]
\centering
\begin{tabular}{|c|c|c|}
\hline
\textbf{Uzel} & \textbf{Stupeň vrcholu} & \textbf{Meziessí} \\
\hline
123 & 15 & 0.234 \\
456 & 14 & 0.198 \\
789 & 13 & 0.185 \\
101 & 12 & 0.162 \\
112 & 11 & 0.145 \\
\hline
\end{tabular}
\caption{Top 5 vrcholů podle stupně a meziessí.}
\label{tab:top-nodes}
\end{table}

\subsection{Komunity a ego-sítě}
Pomocí algoritmu v Gephi byly identifikovány 3 hlavní komunity v síti, které jsou znázorněny na obrázku~\ref{fig:network-visualization}. Dále byly zkonstruovány ego-sítě kolem 5 klíčových vrcholů.

\begin{table}[H]
\centering
\begin{tabular}{|c|c|}
\hline
\textbf{ID Komunity} & \textbf{Velikost} \\
\hline
1 & 120 \\
2 & 89 \\
3 & 78 \\
\hline
\end{tabular}
\caption{Velikosti tří hlavních komunit.}
\label{tab:communities}
\end{table}

\newpage

\subsection{Vizualizace sítě}
Na obrázku~\ref{fig:network-visualization} je vizualizace vyčištěné sítě s komunitami, obarvená podle příslušnosti k jednotlivým komunitám. Síť obsahuje 4 hlavní komunity a 5 významných ego-sítí, které jsou propojeny významnými vrcholy.

\begin{figure}[H]
\centering
\includegraphics[width=0.8\textwidth]{enron.png}
\caption{Vizualizace vyčištěné sítě s komunitami a ego-sítěmi.}
\label{fig:network-visualization}
\end{figure}

\newpage

\section{Závěr}
V této analýze jsem:
\begin{itemize}
    \item Zkonstruoval síť z datasetu \texttt{Email-Enron}.
    \item Provedl prořídnutí sítě a odstranil nevýznamné hrany.
    \item Identifikoval klíčové vrcholy a analyzoval jejich význam na základě stupně a meziessí.
    \item Detekoval 3 hlavní komunity.
    \item Zkonstruoval a analyzoval 5 ego-sítí kolem významných vrcholů.
    \item Vizualizoval síť a komunity.
\end{itemize}

Vyčištěná síť byla uložena ve formátu \texttt{.gexf} pro další zpracování.

\end{document}
