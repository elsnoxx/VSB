\documentclass[10pt,a4paper]{article}
\usepackage[left=20mm,right=30mm,top=30mm,bottom=35mm]{geometry} 
\usepackage[table,xcdraw]{xcolor}
\usepackage{graphicx}
\usepackage{float}





\begin{document}
\begin{titlepage}
    \centering
    \vspace*{2cm} % Vložení vertikální mezery nahoře
    
    
    
    \vspace{2cm} % Další mezera mezi tabulkou a případným textem
    
    \textbf{\Huge Počítačové sítě} \\[1cm]
    \textbf{\Large Projekt} \\[2cm]
    
    
    \vfill % Posune další text dolů (na konec stránky)
    
    \begin{tabular}{|p{5cm}|p{10cm}|}
        \hline
        \rowcolor[HTML]{D9E1F2} \textbf{Předmět:} & \textbf{Počítačové sítě} \\ \hline
        \textbf{Autoři:} & Richard Ficek (FIC0024) \newline Trombik Daniel (TRO0090) \\ \hline
        \textbf{Datum:} & \today \\ \hline
    \end{tabular}
    
\end{titlepage}

\newpage

\tableofcontents

\newpage



\section{Zadání}
Navrhněte a zdokumentujte konfiguraci podnikové sítě připojené do Internetu. Řešení po částech realizujte, otestujte a odevzdejte elektronicky konfigurace a protokoly, dokládající funkčnost způsobem, stanoveným cvičícím/tutorem

\vspace{2cm}

\subsection*{Přidělené rozsahy a parametry}
\begin{table}[htbp]
\centering
\begin{tabular}{|l|l|}
    \hline
    \rowcolor[HTML]{D9E1F2}\textbf{Firma:}                         & phohr \\ \hline
    \textbf{Číslo topologie:}               & H \\ \hline  
    \textbf{Čísla VLAN:}                    & VLAN A=138, VLAN B=178, VLAN C=218 \\ \hline  
    \textbf{Počty stanic na segmentech:}    & VLAN A=250, VLAN B=141 \\ \hline  
    \textbf{Rozsah veřejných adres:}        & 44.118.72.0/21 \\ \hline
    \textbf{Rozsah privátních adres:}       & 10.191.73.0/24 \\ \hline
    \textbf{Rozsah IPv6 adres:}             & 2002:a6a9:dac3::/48 \\ \hline
    \textbf{Zvláštní segmenty:} \newline    & NAT: VLAN A, DNS: VLAN B (PC1B), \\
                                            & DHCP: VLAN B, T: VLAN B, N: VLAN B \\ \hline
    \textbf{NAT pool:}                      & 31 \\ \hline
    \textbf{Směrovací protokol:}            & RIP \\ \hline
\end{tabular}  
\label{tab:address}
\end{table}

\subsection*{Schéma}
\begin{figure}[htbp]
    \centering
    \includegraphics[width=0.8\textwidth]{Topologie.png}
    \caption{Topologie podnikové sítě.}
    \label{fig:topologie}
\end{figure}

% \subsection*{Topologie}
% \begin{figure}[htbp]
%     \centering
%     \includegraphics[width=1.5\textwidth]{topologieL3.png}
%     \caption{L3 topologie}
%     \label{fig:topologie}
% \end{figure}


\newpage


\section{Adresní plán a konfigurace}

\subsection{Adresní plán}



\begin{table}[htbp]
\small
\caption{Veřejné IPv4 adresy}
\centering
\begin{tabular}{|l|l|l|l|l|}
    \hline
    \textbf{Název sítě} & \textbf{Síťová adresa} & \textbf{První adresa} & \textbf{Poslední adresa} & \textbf{Broadcast} \\ \hline
    VLAN B              & 44.118.72.0/24        & 44.118.72.1           & 44.118.72.254            & 44.118.72.255      \\ \hline
    NAT pool            & 44.118.73.0/27        & 44.118.73.1           & 44.118.73.30             & 44.118.73.31       \\ \hline
    R1 - R2             & 44.118.73.32/30       & 44.118.73.33          & 44.118.73.34             & 44.118.73.35       \\ \hline
    R1 - R3             & 44.118.73.36/30       & 44.118.73.37          & 44.118.73.38             & 44.118.73.39       \\ \hline
    R2 - R3             & 44.118.73.40/30       & 44.118.73.41          & 44.118.73.42             & 44.118.73.43       \\ \hline
\end{tabular}
\label{tab:ipv4_adresni_plan}
\end{table}

\begin{table}[htbp]
    \small
    \caption{Privatní IPv4 adresy}
    \centering
    \begin{tabular}{|l|l|l|l|l|l|}
        \hline
        \textbf{Název sítě}     & \textbf{Síťová adresa}   & \textbf{První adresa}       & \textbf{Poslední adresa}      & \textbf{Broadcast} \\ \hline
        VLAN A               & 10.191.73.0/24                     & 10.191.73.1     & 10.191.73.254         & 10.191.73.255      \\ \hline
    \end{tabular}
    \label{tab:ipv4_adresni_plan}
\end{table}

\begin{table}[htbp]
    \small
    \caption{ISP adresy}
    \centering
    \begin{tabular}{|l|l|l|l|l|l|}
        \hline
        \textbf{Název sítě}     & \textbf{Síťová adresa}   & \textbf{První adresa}       & \textbf{Poslední adresa}      & \textbf{Broadcast} \\ \hline
        VLAN C               & 10.0.0.0/30                    & 10.0.0.1     & 10.0.0.2          & 10.0.0.3      \\ \hline
    \end{tabular}
    \label{tab:ipv4_adresni_plan}
\end{table}

\begin{table}[H]
\small
\caption{IPv6 adresy}
\centering
\begin{tabular}{|l|l|l|l|}
    \hline
    \textbf{Název sítě} & \textbf{Síťová adresa}         & \textbf{První adresa}              & \textbf{Poslední adresa}                \\ \hline
    VLAN A              & 2002:a6a9:dac3::/64           & 2002:a6a9:dac3::1                 & 2002:a6a9:dac3::ffff:ffff:ffff:ffff    \\ \hline
    VLAN C              & 2002:a6a9:dac3:1::/64         & 2002:a6a9:dac3:1::1               & 2002:a6a9:dac3:1::ffff:ffff:ffff:ffff  \\ \hline
    VLAN B              & 2002:a6a9:dac3:2::/64         & 2002:a6a9:dac3:2::1               & 2002:a6a9:dac3:2::ffff:ffff:ffff:ffff  \\ \hline
    NAT pool            & 2002:a6a9:dac3:3::/64         & 2002:a6a9:dac3:3::1               & 2002:a6a9:dac3:3::ffff:ffff:ffff:ffff  \\ \hline
    R1 - R2             & 2002:a6a9:dac3:4::/64         & 2002:a6a9:dac3:4::1               & 2002:a6a9:dac3:4::ffff:ffff:ffff:ffff  \\ \hline
    R1 - R3             & 2002:a6a9:dac3:5::/64         & 2002:a6a9:dac3:5::1               & 2002:a6a9:dac3:5::ffff:ffff:ffff:ffff  \\ \hline
    R2 - R3             & 2002:a6a9:dac3:6::/64         & 2002:a6a9:dac3:6::1               & 2002:a6a9:dac3:6::ffff:ffff:ffff:ffff  \\ \hline
\end{tabular}
\label{tab:ipv6_adresni_plan}
\end{table}



\begin{figure}[H]  
    \centering
    \includegraphics[width=1\textwidth]{topologieL3-marked.png}
    \caption{L3 topologie marked}
\end{figure}

\newpage

\subsection{Konfigurace adresace VLAN}

\subsubsection*{VLAN A}


\begin{table}[htbp]
    \centering
    \caption{Adresace pro VLAN A (138)}
    \begin{tabular}{|l|c|}
        \hline
         \textbf{Počet adres:}             & 250 adres                         \\ \hline
        \textbf{Maska:}                    & /24                  \\ \hline
        \textbf{Síťová adresa:}            & 10.191.73.0                      \\ \hline
        \textbf{Rozsah použitelných adres:}& 10.191.73.1 – 10.191.73.254        \\ \hline
        \textbf{Brána:}                    & 10.191.73.1                     \\ \hline
        \textbf{Broadcast:}                & 10.191.73.255                    \\ \hline
    \end{tabular}
    \label{tab:vlan_a_nat}
\end{table} 

\subsubsection*{VLAN B}

\begin{table}[htbp]
    \centering
    \caption{Adresace pro VLAN B (178)}
    \begin{tabular}{|l|c|}
        \hline
         \textbf{Počet adres:}             & 141 adres                         \\ \hline
        \textbf{Maska:}                    & /24                              \\ \hline
        \textbf{Síťová adresa:}            & 44.118.72.0                      \\ \hline
        \textbf{Rozsah použitelných adres:}& 44.118.72.1 - 44.118.72.255        \\ \hline
        \textbf{Brána:}                    & 44.118.72.1                      \\ \hline
        \textbf{Broadcast:}                & 44.118.72.255               \\ \hline
    \end{tabular}
    \label{tab:vlan_b_internal}
\end{table}

\subsubsection*{VLAN C}

\begin{table}[H]
    \centering
    \caption{Adresace pro VLAN C (218)  }
    \begin{tabular}{|l|c|}
        \hline
        \textbf{Maska:}                    & /30                          \\ \hline
        \textbf{Síťová adresa:}            & 10.0.0.0                      \\ \hline
        \textbf{Rozsah použitelných adres:}& 10.0.0.1 - 10.0.0.2        \\ \hline
        \textbf{Brána:}                    & 10.0.0.1                      \\ \hline
        \textbf{Broadcast:}                & 10.0.0.3             \\ \hline
    \end{tabular}
    \label{tab:vlan_b_internal}
\end{table}

\subsubsection*{NAT Pool}


\begin{table}[htbp]
    \centering
    \caption{NAT Pool}
    \begin{tabular}{|l|c|}
        \hline
        \textbf{Počet adres:}             & 31 adres                         \\ \hline
        \textbf{Maska:}                   & /27                              \\ \hline
        \textbf{Síťová adresa:}           & 44.118.73.0                      \\ \hline
        \textbf{Rozsah použitelných adres:}& 44.118.73.1–44.118.73.30         \\ \hline
        \textbf{Broadcast:}               & 44.118.73.31                     \\ \hline
    \end{tabular}
    \label{tab:nat_pool}
\end{table} 



\newpage

\section{Nastavení síťových prvků}

\begin{figure}[H]  
    \centering
    \includegraphics[width=0.8\textwidth]{final-packet-tracer.png}
    \caption{Výsledný náhled z Packet Traceru}
\end{figure}

\newpage


\subsection{Router R1}
\begin{verbatim}
enable
configure terminal
no ip domain-lookup
ip classless

interface FastEthernet0/0
description ethernetového rozhraní pro NAT a ISP
ip address 10.0.0.2 255.255.255.252
ip nat outside
no shutdown
exit

interface serial 0/0/0
description sériového rozhraní mezi R1 a R2
ip address 44.118.73.33 255.255.255.252
ip nat inside
clock rate 64000
no shutdown
exit

interface serial 0/0/1
description sériového rozhraní mezi R1 a R3
ip address 44.118.73.37 255.255.255.252
ip nat inside
clock rate 64000
no shutdown
exit

ip route 0.0.0.0 0.0.0.0 FastEthernet0/0

! RIP směrování
router rip
network 44.118.73.32
network 44.118.73.36
network 10.0.0.0
no auto-summary
passive-interface FastEthernet0/0
default-information originate
version 2
exit
ip nat pool NATPOOL31 44.118.73.0 44.118.73.30 netmask 255.255.255.224
access-list 1 permit 10.191.73.0 0.0.0.255
ip nat inside source list 1 pool NATPOOL31 overload
end
\end{verbatim}


\subsection{Router R2}
\begin{verbatim}
enable
configure terminal
no ip domain-lookup
ip classless
interface serial 0/0/1
description sériového rozhraní mezi R2 a R1
ip address 44.118.73.34 255.255.255.252
no shutdown
exit

interface serial 0/0/0
description sériového rozhraní mezi R2 a R3
ip address 44.118.73.41 255.255.255.252
clock rate 64000
no shutdown
exit

interface fa0/0
description ethernetového rozhraní VLAN B k SW 1
ip address 44.118.72.1 255.255.255.0
no shutdown
exit

! RIP směrování
router rip
version 2
network 44.118.72.0
network 44.118.73.32
network 44.118.73.40
no auto-summary
exit
end
\end{verbatim}

\subsection{Router R3}
\begin{verbatim}
enable
configure terminal
no ip domain-lookup
ip classless
interface serial 0/0/0
description sériového rozhraní mezi R3 a R1
ip address 44.118.73.38 255.255.255.252
no shutdown
exit

interface serial 0/0/1
description sériového rozhraní mezi R3 a R2
ip address 44.118.73.42 255.255.255.252
no shutdown
exit

interface fa0/0
description ethernetového rozhraní VLAN A k SW 2
ip address 10.191.73.1 255.255.255.0
no shutdown
exit

! RIP směrování
router rip
version 2
network 10.191.73.0
network 44.118.73.36
network 44.118.73.40
no auto-summary
exit
end
\end{verbatim}

\subsection{Switch SW1}
\begin{verbatim}
enable
configure terminal
! Vytvoření VLAN
vlan 138
name VLAN_A
vlan 178
name VLAN_B
vlan 218
name VLAN_C
! Přiřazení portů
interface fa0/1
description port ve VLAN A
switchport mode access
switchport access vlan 138
exit
interface fa0/2
description port ve VLAN B
switchport mode access
switchport access vlan 178
exit
interface fa0/3
description port ve VLAN B k R2
switchport mode access
switchport access vlan 178
exit
interface fa0/4
description port ve VLAN C k ISP
switchport mode access
switchport access vlan 218
exit
! Nastavení trunk portu (pro SW2)
interface gi0/1
description trunk linka k SW 2
switchport mode trunk
switchport trunk allowed vlan 138,178,218
no shutdown
exit
end
\end{verbatim}


\subsection{Switch SW2}
\begin{verbatim}
enable
configure terminal
! Vytvoření VLAN
vlan 138
name VLAN_A
vlan 178
name VLAN_B
vlan 218
name VLAN_C
! Přiřazení portů
interface  fa0/1
description port ve VLAN A k R3
switchport mode access
switchport access vlan 138
exit
interface  fa0/2
description port ve VLAN A k PC2A
switchport mode access
switchport access vlan 138
exit
interface  fa0/3
description port ve VLAN B k PC2B
switchport mode access
switchport access vlan 178
exit
interface fa0/4
description port ve VLAN C k R1
switchport mode access
switchport access vlan 218
exit
interface gi0/1
description trunk linka k SW 1
switchport mode trunk
switchport trunk allowed vlan 138,178,218
no shutdown
exit
end

\end{verbatim}

\newpage
\section{Konfigurace DNS a DHCP}

\subsection{DHCP}

Konfigurace DHCP pro router 3, pro privátní adrresy.

\subsection*{Router R2}
\begin{verbatim}
ip dhcp excluded-address 10.191.73.1 10.191.73.10
ip dhcp pool 138
 network 10.191.73.0 255.255.255.0
 default-router 10.191.73.1
 dns-server 10.191.73.10 50.0.0.12
 domain-name phohr.isp.cz

\end{verbatim}
\begin{figure}[H]  
    \centering
    \includegraphics[width=0.8\textwidth]{DHCP-PC2A.png}
    \caption{DHCP nastavení u PC2A}
\end{figure}

\newpage

\subsection{DNS}


\subsection*{Server DNS}
\subsubsection*{Soubor \texttt{named.conf}}
\begin{verbatim}
options {
    directory "/var/cache/bind";
    recursion yes;
};

zone "." {
    type hint;
    file "/etc/bind/db.root";
};

zone "phohr.isp.cz" {
    type master;
    file "/etc/bind/db.phohr";
};

zone "73.191.10.in-addr.arpa" {
    type master;
    file "/etc/bind/db.phohr.rev";
};

zone "72.118.44.in-addr.arpa" {
    type master;
    file "/etc/bind/db.phohr.public.rev";
};

zone "localhost" {
    type master;
    file "/etc/bind/db.local";
};

zone "127.in-addr.arpa" {
    type master;
    file "/etc/bind/db.127";
};

zone "0.in-addr.arpa" {
    type master;
    file "/etc/bind/db.0";
};
\end{verbatim}

\subsubsection*{Soubor \texttt{db.phohr}}
\begin{verbatim}
$ORIGIN isp.cz.
$TTL 604800

phohr IN SOA ns.phohr.isp.cz. admin.phohr.isp.cz. (
    2024121401 ; Serial
    604800     ; Refresh
    86400      ; Retry
    2419200    ; Expire
    604800     ; Minimum TTL
)

$ORIGIN phohr.isp.cz.
NS ns
ns A 10.191.73.10
TXT "Primární DNS server"

; VLAN A
R-1-e1 A 10.191.73.1
R-2-e1 A 10.191.73.2
R-3-e1 A 10.191.73.3

; VLAN B
PC1B A 10.191.73.11
DHCP A 10.191.73.12
T A 10.191.73.13
N A 10.191.73.14

; VLAN C
C1 A 10.191.73.20
C2 A 10.191.73.21
\end{verbatim}

\subsubsection*{Soubor \texttt{db.phohr.rev}}
\begin{verbatim}
$ORIGIN 73.191.10.in-addr.arpa.
$TTL 604800

@ IN SOA ns.phohr.isp.cz. admin.phohr.isp.cz. (
    2024121401 ; Serial
    604800     ; Refresh
    86400      ; Retry
    2419200    ; Expire
    604800     ; Minimum TTL
)

@ IN NS ns.phohr.isp.cz.

1 PTR R-1-e1.phohr.isp.cz.
2 PTR R-2-e1.phohr.isp.cz.
3 PTR R-3-e1.phohr.isp.cz.
11 PTR PC1B.phohr.isp.cz.
12 PTR DHCP.phohr.isp.cz.
13 PTR T.phohr.isp.cz.
14 PTR N.phohr.isp.cz.
20 PTR C1.phohr.isp.cz.
21 PTR C2.phohr.isp.cz.
\end{verbatim}

\subsubsection*{Soubor \texttt{db.phohr.public.rev}}
\begin{verbatim}
$ORIGIN 72.118.44.in-addr.arpa.
$TTL 604800

@ IN SOA ns.phohr.isp.cz. admin.phohr.isp.cz. (
    2024121401 ; Serial
    604800     ; Refresh
    86400      ; Retry
    2419200    ; Expire
    604800     ; Minimum TTL
)

@ IN NS ns.phohr.isp.cz.

1 PTR R-1-e1.phohr.isp.cz.
2 PTR R-2-e1.phohr.isp.cz.
3 PTR R-3-e1.phohr.isp.cz.
\end{verbatim}
\begin{figure}[H]  
    \centering
    \includegraphics[width=0.8\textwidth]{DNS-PC1A.png}
    \caption{DHCP nastavení u PC2A}
\end{figure}


\newpage

\section{Zabezpečení sítě - ACL}

\subsection{Nastavení ACL}
\begin{verbatim}

access-list 100 permit tcp host 40.0.0.11 eq telnet 44.118.72.0 0.0.0.255 established
access-list 100 permit tcp any eq www any established
access-list 100 permit udp any eq 53 any gt 1
access-list 100 permit tcp any eq 53 any established
access-list 100 permit udp any host 44.118.72.61 eq 53
access-list 100 permit tcp any host 44.118.72.61 eq 53
access-list 100 permit icmp any host 44.118.72.61 echo
access-list 100 permit icmp any any echo-reply
access-list 100 permit icmp any host 44.118.72.1 echo
access-list 100 deny icmp any any echo
access-list 100 deny ip 44.118.72.0 0.0.0.255 any
access-list 101 permit tcp 44.118.72.0 0.0.0.255 host 40.0.0.11 eq telnet
access-list 101 deny tcp 44.118.72.0 0.0.0.63 host 30.0.0.10 eq www
access-list 101 permit tcp any any eq www
access-list 101 permit udp any any eq 53
access-list 101 permit tcp any any eq 53
access-list 101 permit udp host 44.118.72.61 eq 53 any gt 1
access-list 101 permit tcp host 44.118.72.61 eq 53 any gt 1
access-list 101 permit icmp any any echo
access-list 101 deny ip 10.0.0.0 0.255.255.255 any
access-list 101 deny ip 172.16.0.0 0.15.255.255 any
access-list 101 deny ip 192.168.0.0 0.0.255.255 any


! Aplikace ACL na rozhraní vedoucí k ISP
interface FastEthernet0/0
 ip access-group 100 in
 ip access-group 101 out

\end{verbatim}

\subsection{Otestování pravidla pro Telnet (příchozí i odchozí)}

\subsubsection*{Ping na DNS server}
\begin{figure}[H]  
    \centering
    \includegraphics[width=0.5\textwidth]{ping-DNS-acl.png}
    \caption{Ping na DNS server}
\end{figure}

\subsubsection*{Ping na IP R1 (10.0.0.2)}
\begin{figure}[H]  
    \centering
    \includegraphics[width=0.5\textwidth]{ping-R1.png}
    \caption{Ping na IP R1 (10.0.0.2}
\end{figure}

\subsubsection*{Ping na WWW server}
\begin{figure}[H]  
    \centering
    \includegraphics[width=0.5\textwidth]{ping-WWW.png}
    \caption{Ping na WWW server}
\end{figure}

\subsubsection*{Připojení na WWW server}
\begin{figure}[H]  
    \centering
    \includegraphics[width=0.5\textwidth]{http-WWW.png}
    \caption{Připojení na WWW server}
\end{figure}




\section{Doplnění chybějících požadavků}

\subsection{IP záznamy}
\begin{figure}[H]  
    \centering
    \includegraphics[width=0.8\textwidth]{IP-configuratio.png}
    \caption{IPv4 a IPv6 nastavení}
\end{figure}

\subsection{Překlad NAT}

\begin{figure}[H]  
    \centering
    \includegraphics[width=0.8\textwidth]{NAT-translation.png}
    \caption{IPv4 a IPv6 nastavení}
\end{figure}

\subsection{Ping a tracert na server poskytovatele}
\begin{figure}[H]  
    \centering
    \includegraphics[width=0.8\textwidth]{tracert-ping.png}
    \caption{Tracert na server poskytovatele}
\end{figure}


\newpage

\subsection{CDP neighbors}

K všem routerům a switchům bylo přidáno nastavení cdp neigbours.

\subsubsection*{R1}
\begin{verbatim}
Capability Codes: R - Router, T - Trans Bridge, B - Source Route Bridge
                  S - Switch, H - Host, I - IGMP, r - Repeater, P - Phone
Device ID    Local Intrfce   Holdtme    Capability   Platform    Port ID
Switch       Fas 0/0          165            S       2960        Fas 0/4
Router       Ser 0/0/0        143            R       C2800       Ser 0/0/1
Router       Ser 0/0/1        158            R       C2800       Ser 0/0/0
\end{verbatim}

\subsubsection*{R2}
\begin{verbatim}
Capability Codes: R - Router, T - Trans Bridge, B - Source Route Bridge
                  S - Switch, H - Host, I - IGMP, r - Repeater, P - Phone
Device ID    Local Intrfce   Holdtme    Capability   Platform    Port ID
Router       Ser 0/0/0        166            R       C2800       Ser 0/0/1
Switch       Fas 0/0          173            S       2960        Fas 0/3
Router       Ser 0/0/1        173            R       C2800       Ser 0/0/0
\end{verbatim}

\subsubsection*{R3}
\begin{verbatim}
Capability Codes: R - Router, T - Trans Bridge, B - Source Route Bridge
                  S - Switch, H - Host, I - IGMP, r - Repeater, P - Phone
Device ID    Local Intrfce   Holdtme    Capability   Platform    Port ID
Switch       Fas 0/0          140            S       2960        Fas 0/1
Router       Ser 0/0/0        140            R       C2800       Ser 0/0/1
Router       Ser 0/0/1        178            R       C2800       Ser 0/0/0
\end{verbatim}

\subsubsection*{SW1}
\begin{verbatim}
Capability Codes: R - Router, T - Trans Bridge, B - Source Route Bridge
                  S - Switch, H - Host, I - IGMP, r - Repeater, P - Phone
Device ID    Local Intrfce   Holdtme    Capability   Platform    Port ID
Switch       Gig 0/1          141            S       2960        Gig 0/1
RISP         Fas 0/4          141            R       C2800       Fas 0/0
Router       Fas 0/3          179            R       C2800       Fas 0/0
\end{verbatim}

\subsubsection*{SW2}
\begin{verbatim}
Capability Codes: R - Router, T - Trans Bridge, B - Source Route Bridge
                  S - Switch, H - Host, I - IGMP, r - Repeater, P - Phone
Device ID    Local Intrfce   Holdtme    Capability   Platform    Port ID
Switch       Gig 0/1          120            S       2960        Gig 0/1
Router       Fas 0/4          120            R       C2800       Fas 0/0
Router       Fas 0/1          173            R       C2800       Fas 0/0
\end{verbatim}


\newpage

\subsection{Data směrovacího protokolu a počet naučených záznamů}

\subsubsection*{R1}
\begin{verbatim}
Gateway of last resort is 0.0.0.0 to network 0.0.0.0

     10.0.0.0/8 is variably subnetted, 2 subnets, 2 masks
C       10.0.0.0/30 is directly connected, FastEthernet0/0
R       10.191.73.0/24 [120/1] via 44.118.73.38, 00:00:23, Serial0/0/1
     44.0.0.0/8 is variably subnetted, 4 subnets, 2 masks
R       44.118.72.0/24 [120/1] via 44.118.73.34, 00:00:08, Serial0/0/0
C       44.118.73.32/30 is directly connected, Serial0/0/0
C       44.118.73.36/30 is directly connected, Serial0/0/1
R       44.118.73.40/30 [120/1] via 44.118.73.38, 00:00:23, Serial0/0/1
                        [120/1] via 44.118.73.34, 00:00:08, Serial0/0/0
S*   0.0.0.0/0 is directly connected, FastEthernet0/0
\end{verbatim}

\subsubsection*{R2}
\begin{verbatim}
Gateway of last resort is 44.118.73.33 to network 0.0.0.0

     10.0.0.0/8 is variably subnetted, 2 subnets, 2 masks
R       10.0.0.0/30 [120/1] via 44.118.73.33, 00:00:25, Serial0/0/1
R       10.191.73.0/24 [120/1] via 44.118.73.42, 00:00:22, Serial0/0/0
     44.0.0.0/8 is variably subnetted, 4 subnets, 2 masks
C       44.118.72.0/24 is directly connected, FastEthernet0/0
C       44.118.73.32/30 is directly connected, Serial0/0/1
R       44.118.73.36/30 [120/1] via 44.118.73.42, 00:00:22, Serial0/0/0
                        [120/1] via 44.118.73.33, 00:00:25, Serial0/0/1
C       44.118.73.40/30 is directly connected, Serial0/0/0
R*   0.0.0.0/0 [120/1] via 44.118.73.33, 00:00:25, Serial0/0/1
\end{verbatim}

\subsubsection*{R3}
\begin{verbatim}
Gateway of last resort is 44.118.73.37 to network 0.0.0.0

     10.0.0.0/8 is variably subnetted, 2 subnets, 2 masks
R       10.0.0.0/30 [120/1] via 44.118.73.37, 00:00:26, Serial0/0/0
C       10.191.73.0/24 is directly connected, FastEthernet0/0
     44.0.0.0/8 is variably subnetted, 4 subnets, 2 masks
R       44.118.72.0/24 [120/1] via 44.118.73.41, 00:00:05, Serial0/0/1
R       44.118.73.32/30 [120/1] via 44.118.73.37, 00:00:26, Serial0/0/0
                        [120/1] via 44.118.73.41, 00:00:05, Serial0/0/1
C       44.118.73.36/30 is directly connected, Serial0/0/0
C       44.118.73.40/30 is directly connected, Serial0/0/1
R*   0.0.0.0/0 [120/1] via 44.118.73.37, 00:00:26, Serial0/0/0
\end{verbatim}

\newpage

\subsection{Konfigurace RIP}

\subsubsection*{R1}
\begin{verbatim}
Routing Protocol is "rip"
Sending updates every 30 seconds, next due in 18 seconds
Invalid after 180 seconds, hold down 180, flushed after 240
Outgoing update filter list for all interfaces is not set
Incoming update filter list for all interfaces is not set
Redistributing: rip
Default version control: send version 2, receive 2
  Interface             Send  Recv  Triggered RIP  Key-chain
  Serial0/0/1           22
  Serial0/0/0           22
Automatic network summarization is not in effect
Maximum path: 4
Routing for Networks:
	10.0.0.0
	44.0.0.0
Passive Interface(s):
	FastEthernet0/0
Routing Information Sources:
	Gateway         Distance      Last Update
	44.118.73.38         120      00:00:08
	44.118.73.34         120      00:00:15
Distance: (default is 120)
\end{verbatim}

\subsubsection*{R2}
\begin{verbatim}
Routing Protocol is "rip"
Sending updates every 30 seconds, next due in 0 seconds
Invalid after 180 seconds, hold down 180, flushed after 240
Outgoing update filter list for all interfaces is not set
Incoming update filter list for all interfaces is not set
Redistributing: rip
Default version control: send version 2, receive 2
  Interface             Send  Recv  Triggered RIP  Key-chain
  FastEthernet0/0       22
  Serial0/0/0           22
  Serial0/0/1           22
Automatic network summarization is not in effect
Maximum path: 4
Routing for Networks:
	44.0.0.0
Passive Interface(s):
Routing Information Sources:
	Gateway         Distance      Last Update
	44.118.73.33         120      00:00:19
	44.118.73.42         120      00:00:20
Distance: (default is 120)
\end{verbatim}

\subsubsection*{R3}
\begin{verbatim}
Routing Protocol is "rip"
Sending updates every 30 seconds, next due in 4 seconds
Invalid after 180 seconds, hold down 180, flushed after 240
Outgoing update filter list for all interfaces is not set
Incoming update filter list for all interfaces is not set
Redistributing: rip
Default version control: send version 2, receive 2
  Interface             Send  Recv  Triggered RIP  Key-chain
  FastEthernet0/0       22
  Serial0/0/0           22
  Serial0/0/1           22
Automatic network summarization is not in effect
Maximum path: 4
Routing for Networks:
	10.0.0.0
	44.0.0.0
Passive Interface(s):
Routing Information Sources:
	Gateway         Distance      Last Update
	44.118.73.37         120      00:00:24
	44.118.73.41         120      00:00:05
Distance: (default is 120)
\end{verbatim}


\subsection{Úprava nastavení směrovačů pro IPv6}

\subsubsection*{R1}
\begin{verbatim}
enable
configure terminal
ipv6 unicast-routing
no ip domain-lookup
ip classless

interface FastEthernet0/0
description ethernetového rozhraní pro NAT a ISP
ip address 10.0.0.2 255.255.255.252
ipv6 address 2002:a6a9:dac3:1::1/64
ipv6 enable
ip nat outside
no shutdown
exit

interface serial 0/0/0
description sériového rozhraní mezi R1 a R2
ip address 44.118.73.33 255.255.255.252
ipv6 address 2002:a6a9:dac3:4::1/64
ipv6 enable
ip nat inside
clock rate 64000
no shutdown
exit

interface serial 0/0/1
description sériového rozhraní mezi R1 a R3
ip address 44.118.73.37 255.255.255.252
ipv6 address 2002:a6a9:dac3:5::1/64
ipv6 enable
ip nat inside
clock rate 64000
no shutdown
exit

ip route 0.0.0.0 0.0.0.0 FastEthernet0/0
ipv6 route ::/0 FastEthernet0/0

! RIP směrování
router rip
network 44.118.73.32
network 44.118.73.36
network 10.0.0.0
no auto-summary
passive-interface FastEthernet0/0
default-information originate
version 2
exit
ipv6 router rip RIPNG
interface FastEthernet0/0
ipv6 rip RIPNG enable
exit
ip nat pool NATPOOL31 44.118.73.0 44.118.73.30 netmask 255.255.255.224
access-list 1 permit 10.191.73.0 0.0.0.255
ip nat inside source list 1 pool NATPOOL31 overload
end

\end{verbatim}

\subsubsection*{R2}
\begin{verbatim}
enable
configure terminal
ipv6 unicast-routing
no ip domain-lookup
ip classless
interface serial 0/0/1
description sériového rozhraní mezi R2 a R1
ip address 44.118.73.34 255.255.255.252
ipv6 address 2002:a6a9:dac3:4::2/64
ipv6 enable
no shutdown
exit

interface serial 0/0/0
description sériového rozhraní mezi R2 a R3
ip address 44.118.73.41 255.255.255.252
ipv6 address 2002:a6a9:dac3:6::1/64
ipv6 enable
clock rate 64000
no shutdown
exit

interface fa0/0
description ethernetového rozhraní VLAN B k SW 1
ip address 44.118.72.1 255.255.255.0
ipv6 address 2002:a6a9:dac3:2::1/64
ipv6 enable
no shutdown
exit

! RIP směrování
router rip
version 2
network 44.118.72.0
network 44.118.73.32
network 44.118.73.40
no auto-summary
exit
end

\end{verbatim}

\subsubsection*{R3}
\begin{verbatim}
enable
configure terminal
ipv6 unicast-routing
no ip domain-lookup
ip classless
interface serial 0/0/0
description sériového rozhraní mezi R3 a R1
ip address 44.118.73.38 255.255.255.252
ipv6 address 2002:a6a9:dac3:5::2/64
ipv6 enable
no shutdown
exit

interface serial 0/0/1
description sériového rozhraní mezi R3 a R2
ip address 44.118.73.42 255.255.255.252
ipv6 address 2002:a6a9:dac3:6::2/64
ipv6 enable
no shutdown
exit

interface fa0/0
description ethernetového rozhraní VLAN A k SW 2
ip address 10.191.73.1 255.255.255.0
ipv6 address 2002:a6a9:dac3::1/64
ipv6 enable
no shutdown
exit

! RIP směrování
router rip
version 2
network 10.191.73.0
network 44.118.73.36
network 44.118.73.40
no auto-summary
exit
end

\end{verbatim}


\subsection{IPv6 routing}
\subsubsection*{R1}
\begin{verbatim}
IPv6 Routing Table - 17 entries
Codes: C - Connected, L - Local, S - Static, R - RIP, B - BGP
       U - Per-user Static route, M - MIPv6
       I1 - ISIS L1, I2 - ISIS L2, IA - ISIS interarea, IS - ISIS summary
       ND - ND Default, NDp - ND Prefix, DCE - Destination, NDr - Redirect
       O - OSPF intra, OI - OSPF inter, OE1 - OSPF ext 1, OE2 - OSPF ext 2
       ON1 - OSPF NSSA ext 1, ON2 - OSPF NSSA ext 2
       D - EIGRP, EX - EIGRP external
S   ::/0 [1/0]
     via ::, FastEthernet0/0
C   2001:DB8:1::/64 [0/0]
     via ::, FastEthernet0/0
L   2001:DB8:1::1/128 [0/0]
     via ::, FastEthernet0/0
C   2001:DB8:2::/64 [0/0]
     via ::, Serial0/0/0
L   2001:DB8:2::1/128 [0/0]
     via ::, Serial0/0/0
C   2001:DB8:3::/64 [0/0]
     via ::, Serial0/0/1
L   2001:DB8:3::1/128 [0/0]
     via ::, Serial0/0/1
R   2001:DB8:4::/64 [120/2]
     via FE80::20C:CFFF:FE6D:2602, Serial0/0/0
C   2002:A6A9:DAC3:1::/64 [0/0]
     via ::, FastEthernet0/0
L   2002:A6A9:DAC3:1::1/128 [0/0]
     via ::, FastEthernet0/0
R   2002:A6A9:DAC3:2::/64 [120/2]
     via FE80::20C:CFFF:FE6D:2602, Serial0/0/0
C   2002:A6A9:DAC3:4::/64 [0/0]
     via ::, Serial0/0/0
L   2002:A6A9:DAC3:4::1/128 [0/0]
     via ::, Serial0/0/0
C   2002:A6A9:DAC3:5::/64 [0/0]
     via ::, Serial0/0/1
L   2002:A6A9:DAC3:5::1/128 [0/0]
     via ::, Serial0/0/1
R   2002:A6A9:DAC3:6::/64 [120/2]
     via FE80::20C:CFFF:FE6D:2602, Serial0/0/0
L   FF00::/8 [0/0]
     via ::, Null0
\end{verbatim}

\subsubsection*{R2}
\begin{verbatim}
IPv6 Routing Table - 16 entries
Codes: C - Connected, L - Local, S - Static, R - RIP, B - BGP
       U - Per-user Static route, M - MIPv6
       I1 - ISIS L1, I2 - ISIS L2, IA - ISIS interarea, IS - ISIS summary
       ND - ND Default, NDp - ND Prefix, DCE - Destination, NDr - Redirect
       O - OSPF intra, OI - OSPF inter, OE1 - OSPF ext 1, OE2 - OSPF ext 2
       ON1 - OSPF NSSA ext 1, ON2 - OSPF NSSA ext 2
       D - EIGRP, EX - EIGRP external
R   2001:DB8:1::/64 [120/2]
     via FE80::20C:85FF:FEB2:B401, Serial0/0/1
C   2001:DB8:2::/64 [0/0]
     via ::, Serial0/0/1
L   2001:DB8:2::2/128 [0/0]
     via ::, Serial0/0/1
C   2001:DB8:3::/64 [0/0]
     via ::, Serial0/0/0
L   2001:DB8:3::2/128 [0/0]
     via ::, Serial0/0/0
C   2001:DB8:4::/64 [0/0]
     via ::, FastEthernet0/0
L   2001:DB8:4::1/128 [0/0]
     via ::, FastEthernet0/0
R   2002:A6A9:DAC3:1::/64 [120/2]
     via FE80::20C:85FF:FEB2:B401, Serial0/0/1
C   2002:A6A9:DAC3:2::/64 [0/0]
     via ::, FastEthernet0/0
L   2002:A6A9:DAC3:2::1/128 [0/0]
     via ::, FastEthernet0/0
C   2002:A6A9:DAC3:4::/64 [0/0]
     via ::, Serial0/0/1
L   2002:A6A9:DAC3:4::2/128 [0/0]
     via ::, Serial0/0/1
R   2002:A6A9:DAC3:5::/64 [120/2]
     via FE80::20C:85FF:FEB2:B401, Serial0/0/1
C   2002:A6A9:DAC3:6::/64 [0/0]
     via ::, Serial0/0/0
L   2002:A6A9:DAC3:6::1/128 [0/0]
     via ::, Serial0/0/0
L   FF00::/8 [0/0]
     via ::, Null0
\end{verbatim}

\subsubsection*{R3}
\begin{verbatim}
IPv6 Routing Table - 7 entries
Codes: C - Connected, L - Local, S - Static, R - RIP, B - BGP
       U - Per-user Static route, M - MIPv6
       I1 - ISIS L1, I2 - ISIS L2, IA - ISIS interarea, IS - ISIS summary
       ND - ND Default, NDp - ND Prefix, DCE - Destination, NDr - Redirect
       O - OSPF intra, OI - OSPF inter, OE1 - OSPF ext 1, OE2 - OSPF ext 2
       ON1 - OSPF NSSA ext 1, ON2 - OSPF NSSA ext 2
       D - EIGRP, EX - EIGRP external
C   2002:A6A9:DAC3::/64 [0/0]
     via ::, FastEthernet0/0
L   2002:A6A9:DAC3::1/128 [0/0]
     via ::, FastEthernet0/0
C   2002:A6A9:DAC3:5::/64 [0/0]
     via ::, Serial0/0/0
L   2002:A6A9:DAC3:5::2/128 [0/0]
     via ::, Serial0/0/0
C   2002:A6A9:DAC3:6::/64 [0/0]
     via ::, Serial0/0/1
L   2002:A6A9:DAC3:6::2/128 [0/0]
     via ::, Serial0/0/1
L   FF00::/8 [0/0]
     via ::, Null0
\end{verbatim}


\end{document}