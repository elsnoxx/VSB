\documentclass[10pt,a4paper]{article}
\usepackage[left=20mm,right=30mm,top=30mm,bottom=35mm]{geometry} 

\usepackage[english]{babel}
\usepackage{longtable}
\usepackage{array}
\usepackage{pdflscape}
\usepackage{xcolor}
\usepackage{graphicx}
\usepackage{fancyhdr}
\usepackage{hyperref}
\usepackage{float}
\hypersetup{
    colorlinks=true,
    linkcolor=blue,
    filecolor=magenta,      
    urlcolor=cyan,
    pdftitle={Project DS II - Hotel},
    pdfpagemode=FullScreen,
}

\pagestyle{fancy}
\fancyhf{}
\renewcommand{\headrulewidth}{0pt}
\rhead{Project TDS I - Hotel}
\lfoot{Richard Ficek, FIC0024}
\rfoot{\thepage}

\title{Project TDS I - Hotel}
\author{Richard Ficek, FIC0024}
\date{\today}

\begin{document}
\maketitle

\newpage

\tableofcontents

\newpage



\section{DD S01 L02}

\texttt{Task: } Explain the difference between the concept of data and information - an example in your project written in English

\bigskip

\textbf{Data} are raw, unprocessed facts without context or meaning. \textbf{Information} is data that has been processed, organized, or interpreted to provide meaning or value.

\medskip

\textbf{Example from the project:}

\begin{itemize}
    \item \textbf{Data:} In the \texttt{Guest} table, a single record such as:
    \begin{quote}
        firstname: ``John'', lastname: ``Smith'', email: ``john.smith@email.com'', birth\_date: ``1985-06-15''
    \end{quote}
    is just a set of data points about a guest.

    Similarly, in the \texttt{Reservation} table, several records might look like:
\begin{description}
    \item[reservation\_id] 1
    \item[guest\_id] 5
    \item[room\_id] 101
    \item[employee\_id] 3
    \item[creation\_date] 2024-04-01
    \item[check\_in\_date] 2024-04-10
    \item[check\_out\_date] 2024-04-15
    \item[payment\_id] 7
    \item[status] \texttt{"confirmed"}
    \item[accommodation\_price] 500.00
\end{description}

    These are just raw data entries about guests and reservations.

    \item \textbf{Information:} If we process the data, we can find that ``Most guests who stayed in the hotel in 2024 were born after 1990,'' or, after analyzing the reservations, we can obtain information such as: \\
    \textit{“In April 2024, the average length of stay for confirmed reservations was 3.5 nights, and the average accommodation price was 475.00. Most reservations were made for mid-April.”} \\
    This information provides insight and helps in decision-making, such as targeting marketing campaigns to a younger audience or optimizing pricing and staffing for busy periods.
\end{itemize}

\begin{figure}[H]
    \centering
    \includegraphics[width=0.3\textwidth]{reservation.png}
    \label{fig:matrix_diagram}
\end{figure}

\newpage

\section{DD S02 L02}

\texttt{Task: } Entities, instances, attributes and identifiers - describe in examples on your project

\bigskip

\begin{itemize}
    \item \textbf{Entity:} An entity is an object or concept about which data is stored. In the hotel project, examples of entities are \texttt{Guest}, \texttt{Reservation}, \texttt{Room}, and \texttt{Employee}.
    
    \item \textbf{Instance:} An instance is a specific occurrence of an entity. For example, a single guest record: \\
    \texttt{guest\_id: 1, firstname: "John", lastname: "Smith", email: "john.smith@email.com", birth\_date: "1985-06-15"} \\
    is an instance of the \texttt{Guest} entity.

    \item \textbf{Attribute:} An attribute is a property or characteristic of an entity. For the \texttt{Reservation} entity, attributes include \texttt{reservation\_id}, \texttt{guest\_id}, \texttt{room\_id}, \texttt{check\_in\_date}, \texttt{status}, and \texttt{accommodation\_price}.

    \item \textbf{Identifier:} An identifier uniquely distinguishes each instance of an entity. For example, \texttt{guest\_id} is the identifier for the \texttt{Guest} entity, and \texttt{reservation\_id} is the identifier for the \texttt{Reservation} entity.
\end{itemize}

\medskip

\textbf{Example:}

\begin{quote}
\texttt{
reservation\_id: 5, guest\_id: 2, room\_id: 201, employee\_id: 1, creation\_date: 2024-05-01, check\_in\_date: 2024-05-10, check\_out\_date: 2024-05-15, payment\_id: 4, status: "confirmed", accommodation\_price: 600.00
}
\end{quote}

In this example:
\begin{itemize}
    \item \texttt{Reservation} is the entity.
    \item The quoted record is an instance of the \texttt{Reservation} entity.
    \item Each field (e.g., \texttt{check\_in\_date}, \texttt{status}) is an attribute.
    \item \texttt{reservation\_id} is the identifier.
\end{itemize}

\begin{figure}[H]
    \centering

    \begin{minipage}[b]{0.45\textwidth}
        \centering
        \includegraphics[width=\textwidth]{reservation.png}
        \caption{Rezervace}
        \label{fig:reservation}
    \end{minipage}
    \hspace{0.05\textwidth}
    \begin{minipage}[b]{0.45\textwidth}
        \centering
        \includegraphics[width=\textwidth]{room.png}
        \caption{Pokoj}
        \label{fig:room}
    \end{minipage}

\end{figure}

\newpage

\section*{Database Tables and Attributes}

\section*{Attributes: Guest}
\begin{longtable}{|>{\raggedright\arraybackslash}p{5cm}|>{\raggedright\arraybackslash}p{9cm}|}
\hline
\textbf{Attribute} & \textbf{Description} \\
\hline
guest\_id & Unique identifier for guest (auto-generated) \\
firstname & Guest's first name \\
lastname & Guest's last name \\
email & Guest's email address \\
phone & Guest's phone number \\
birth\_date & Guest's date of birth \\
street & Guest's street address \\
city & Guest's city \\
postal\_code & Guest's postal code \\
country & Guest's country \\
guest\_type & Type of guest (standard, VIP, etc.) \\
registration\_date & Date when guest registered (default: current date) \\
manager\_id & Reference to managing employee \\
notes & Additional notes about the guest \\
\hline
\end{longtable}

\section*{Attributes: Employee}
\begin{longtable}{|>{\raggedright\arraybackslash}p{5cm}|>{\raggedright\arraybackslash}p{9cm}|}
\hline
\textbf{Attribute} & \textbf{Description} \\
\hline
employee\_id & Unique identifier for employee (auto-generated) \\
firstname & Employee's first name \\
lastname & Employee's last name \\
position & Employee's job position \\
street & Employee's street address \\
city & Employee's city \\
postal\_code & Employee's postal code \\
country & Employee's country \\
\hline
\end{longtable}

\section*{Attributes: RoomType}
\begin{longtable}{|>{\raggedright\arraybackslash}p{5cm}|>{\raggedright\arraybackslash}p{9cm}|}
\hline
\textbf{Attribute} & \textbf{Description} \\
\hline
room\_type\_id & Unique identifier for room type (auto-generated) \\
name & Name of the room type \\
bed\_count & Number of beds in this room type \\
\hline
\end{longtable}

\section*{Attributes: Room}
\begin{longtable}{|>{\raggedright\arraybackslash}p{5cm}|>{\raggedright\arraybackslash}p{9cm}|}
\hline
\textbf{Attribute} & \textbf{Description} \\
\hline
room\_id & Unique identifier for room (auto-generated) \\
room\_type\_id & Reference to room type \\
room\_number & Unique room number \\
is\_occupied & Room occupation status (0=free, 1=occupied) \\
\hline
\end{longtable}

\section*{Attributes: Payment}
\begin{longtable}{|>{\raggedright\arraybackslash}p{5cm}|>{\raggedright\arraybackslash}p{9cm}|}
\hline
\textbf{Attribute} & \textbf{Description} \\
\hline
payment\_id & Unique identifier for payment (auto-generated) \\
total\_accommodation & Total accommodation cost \\
total\_expenses & Total additional expenses \\
payment\_date & Date of payment \\
is\_paid & Payment status (0=unpaid, 1=paid) \\
\hline
\end{longtable}

\section*{Attributes: Reservation}
\begin{longtable}{|>{\raggedright\arraybackslash}p{5cm}|>{\raggedright\arraybackslash}p{9cm}|}
\hline
\textbf{Attribute} & \textbf{Description} \\
\hline
reservation\_id & Unique identifier for reservation (auto-generated) \\
guest\_id & Reference to guest making the reservation \\
room\_id & Reference to reserved room \\
employee\_id & Reference to employee handling the reservation \\
creation\_date & Date when reservation was created \\
check\_in\_date & Guest check-in date \\
check\_out\_date & Guest check-out date \\
payment\_id & Reference to payment for this reservation \\
status & Reservation status (confirmed, cancelled, etc.) \\
accommodation\_price & Price for accommodation \\
\hline
\end{longtable}

\section*{Attributes: Service}
\begin{longtable}{|>{\raggedright\arraybackslash}p{5cm}|>{\raggedright\arraybackslash}p{9cm}|}
\hline
\textbf{Attribute} & \textbf{Description} \\
\hline
service\_id & Unique identifier for service (auto-generated) \\
name & Name of the service \\
description & Detailed description of the service \\
\hline
\end{longtable}

\section*{Attributes: ServiceUsage}
\begin{longtable}{|>{\raggedright\arraybackslash}p{5cm}|>{\raggedright\arraybackslash}p{9cm}|}
\hline
\textbf{Attribute} & \textbf{Description} \\
\hline
usage\_id & Unique identifier for service usage (auto-generated) \\
reservation\_id & Reference to reservation using the service \\
service\_id & Reference to service being used \\
quantity & Quantity of service used \\
total\_price & Total price for service usage \\
\hline
\end{longtable}

\newpage

\section*{Attributes: Feedback}
\begin{longtable}{|>{\raggedright\arraybackslash}p{5cm}|>{\raggedright\arraybackslash}p{9cm}|}
\hline
\textbf{Attribute} & \textbf{Description} \\
\hline
feedback\_id & Unique identifier for feedback (auto-generated) \\
guest\_id & Reference to guest providing feedback \\
reservation\_id & Reference to reservation being rated \\
rating & Numerical rating for the stay \\
note & Additional comments from guest \\
feedback\_date & Date when feedback was submitted \\
\hline
\end{longtable}

\section*{Attributes: ServicePriceHistory}
\begin{longtable}{|>{\raggedright\arraybackslash}p{5cm}|>{\raggedright\arraybackslash}p{9cm}|}
\hline
\textbf{Attribute} & \textbf{Description} \\
\hline
sph\_id & Unique identifier for service price history (auto-generated) \\
service\_id & Reference to service \\
price & Price of service during this period \\
valid\_from & Start date of price validity \\
valid\_to & End date of price validity \\
\hline
\end{longtable}

\section*{Attributes: RoomTypePriceHistory}
\begin{longtable}{|>{\raggedright\arraybackslash}p{5cm}|>{\raggedright\arraybackslash}p{9cm}|}
\hline
\textbf{Attribute} & \textbf{Description} \\
\hline
rtph\_id & Unique identifier for room type price history (auto-generated) \\
room\_type\_id & Reference to room type \\
price\_per\_night & Price per night during this period \\
valid\_from & Start date of price validity \\
valid\_to & End date of price validity \\
\hline
\end{longtable}

\newpage

\section{DD S03 L01}

\texttt{Task: } Describe all relations in your database in English, including cardinality and membership obligation - each relation in two sentences (page 10)

\bigskip

\begin{itemize}
    \item \textbf{Guest -- Reservation:} Each guest can have zero or more reservations (1:N). Every reservation must be linked to exactly one guest (mandatory).
    
    \item \textbf{Room -- Reservation:} Each room can be associated with zero or more reservations over time (1:N). Every reservation must be assigned to exactly one room (mandatory).
    
    \item \textbf{Employee -- Reservation:} Each employee can create or manage zero or more reservations (1:N). Every reservation must be linked to exactly one employee (mandatory).
    
    \item \textbf{Payment -- Reservation:} Each payment can be linked to one or more reservations (1:N). Every reservation must have exactly one payment (mandatory).
    
    \item \textbf{RoomType -- Room:} Each room type can be assigned to zero or more rooms (1:N). Every room must have exactly one room type (mandatory).
    
    \item \textbf{Service -- ServiceUsage:} Each service can be used in zero or more service usages (1:N). Every service usage must refer to exactly one service (mandatory).
    
    \item \textbf{Reservation -- ServiceUsage:} Each reservation can have zero or more service usages (1:N). Every service usage must be linked to exactly one reservation (mandatory).
    
    \item \textbf{Guest -- Feedback:} Each guest can provide zero or more feedback entries (1:N). Every feedback must be linked to exactly one guest (mandatory).
    
    \item \textbf{Reservation -- Feedback:} Each reservation can have zero or more feedback entries (1:N). Every feedback must be linked to exactly one reservation (mandatory).
    
    \item \textbf{Service -- ServicePriceHistory:} Each service can have zero or more price history records (1:N). Every price history record must be linked to exactly one service (mandatory).
    
    \item \textbf{RoomType -- RoomTypePriceHistory:} Each room type can have zero or more price history records (1:N). Every price history record must be linked to exactly one room type (mandatory).
\end{itemize}

\bigskip

\newpage

\section{DD S03 L02}

\texttt{Task: } Draw an ER diagram according to conventions

\bigskip

\begin{figure}[H]
    \centering
    \includegraphics[width=0.8\textwidth]{konceptualERD.png}
    \caption{ER Diagram for Hotel Database}
    \label{fig:er_diagram}
\end{figure}


\newpage

\begin{landscape}
\section{DD S30 L04}

\texttt{Task: } Matrix diagram with relationships, draw for your solution 

\bigskip

\setlength{\tabcolsep}{2pt}
\renewcommand{\arraystretch}{1.2}
\begin{longtable}{|l|c|c|c|c|c|c|c|c|c|c|}
\hline
 & \textbf{Guest} & \textbf{Employee} & \textbf{Room} & \textbf{RoomType} & \textbf{Reservation} & \textbf{Payment} & \textbf{Service} & \textbf{ServiceUsage} & \textbf{Feedback} & \textbf{ServicePriceHistory} \\
\hline
\textbf{Guest} &  &  &  &  & 1:N &  &  &  & 1:N &  \\
\hline
\textbf{Employee} &  &  &  &  & 1:N &  &  &  &  &  \\
\hline
\textbf{Room} &  &  &  & 1:N & 1:N &  &  &  &  &  \\
\hline
\textbf{RoomType} &  &  & N:1 &  &  &  &  &  &  & 1:N \\
\hline
\textbf{Reservation} & N:1 & N:1 & N:1 &  &  & 1:1 &  & 1:N & 1:N &  \\
\hline
\textbf{Payment} &  &  &  &  & 1:1 &  &  &  &  &  \\
\hline
\textbf{Service} &  &  &  &  &  &  &  & 1:N &  & 1:N \\
\hline
\textbf{ServiceUsage} &  &  &  &  & N:1 &  & N:1 &  &  &  \\
\hline
\textbf{Feedback} & N:1 &  &  &  & N:1 &  &  &  &  &  \\
\hline
\textbf{ServicePriceHistory} &  &  &  & N:1 &  &  & N:1 &  &  &  \\
\hline
\end{longtable}
\end{landscape}

\newpage

\section{DD S04 L01}

\texttt{Task: } Supertypes and subtypes – define at least one instance of a supertype and a subtype in your project

\bigskip

In the hotel project, an example of a supertype and subtype can be found in the \texttt{Guest} entity. The \texttt{Guest} table has an attribute \texttt{guest\_type}, which can distinguish between different subtypes such as "Regular" and "VIP".

\medskip

\textbf{Supertype:} \texttt{Guest} (stores general information about all guests, regardless of type).

\textbf{Subtypes:}
\begin{itemize}
    \item \texttt{Regular Guest} (a standard guest, e.g., \texttt{guest\_type = "Regular"})
    \item \texttt{VIP Guest} (a guest with VIP status, e.g., \texttt{guest\_type = "VIP"})
\end{itemize}

\textbf{Example instance:}
\begin{quote}
\texttt{guest\_id: 10, firstname: "Alice", lastname: "Brown", email: "alice.brown@email.com", guest\_type: "VIP"}
\end{quote}
This record is an instance of the supertype \texttt{Guest} and the subtype \texttt{VIP Guest}.

\begin{figure}[H]
    \centering
    \includegraphics[width=0.5\textwidth]{guest.png}
    \caption{Guest table}
    \label{fig:er_diagram}
\end{figure}

\newpage

\section{DD S04 L02}

\texttt{Task: } Description of business rules for your project

\bigskip

\begin{itemize}
    \item Each guest must provide a unique email address and basic personal information to register.
    \item A reservation can only be created if the selected room is available for the entire requested period.
    \item Every reservation must be linked to exactly one guest, one room, one employee, and one payment.
    \item Each payment must cover the total accommodation and any additional expenses, and can be marked as paid or unpaid.
    \item Services can be used only by guests with an active reservation, and each service usage must be linked to a reservation.
    \item Feedback can only be submitted by guests who have completed a reservation.
    \item Room prices and service prices can change over time, but each price change must be recorded in the corresponding price history table.
    \item Each room must belong to exactly one room type, and each room type can have multiple rooms.
    \item Only one guest type (Individual or Company) can be assigned to each guest.
    \item Employees are responsible for managing reservations and must be assigned to each reservation.
\end{itemize}

\begin{table}[H]
\caption{Guest Table Attributes}
\begin{tabular}{|l|l|c|c|c|c|c|l|}
    \hline
    \textbf{Name} & \textbf{Data Type} & \textbf{Length} & \textbf{Key} & \textbf{Null} & \textbf{Index} & \textbf{IO} & \textbf{Meaning} \\ \hline
    guest\_id & int & - & P & N & A & - & Guest ID \\ \hline
    firstname & varchar & 100 & - & N & - & - & Guest first name \\ \hline
    lastname & varchar & 100 & - & N & - & - & Guest last name \\ \hline
    email & varchar & 100 & - & N & - & - & Guest email \\ \hline
    phone & varchar & 30 & - & A & - & - & Guest phone number \\ \hline
    birth\_date & date & - & - & N & - & - & Guest birth date \\ \hline
    street & varchar & 255 & - & N & - & - & Street \\ \hline
    city & varchar & 100 & - & N & - & - & City \\ \hline
    postal\_code & char & 10 & - & N & - & - & Postal code \\ \hline
    country & varchar & 100 & - & N & - & - & Country \\ \hline
    guest\_type & varchar & 50 & - & N & - & - & Guest type \\ \hline
    registration\_date & date & - & - & A & - & - & Registration date \\ \hline
    notes & clob & - & - & A & - & - & Notes about guest \\ 
    \hline
\end{tabular}
\label{tab:guest}
\end{table}

\begin{table}[H]
\caption{Employee Table Attributes}
\begin{tabular}{|l|l|c|c|c|c|c|l|}
    \hline
    \textbf{Name} & \textbf{Data Type} & \textbf{Length} & \textbf{Key} & \textbf{Null} & \textbf{Index} & \textbf{IO} & \textbf{Meaning} \\ \hline
    employee\_id & int & - & P & N & A & - & Employee ID \\ \hline
    firstname & varchar & 100 & - & N & - & - & Employee first name \\ \hline
    lastname & varchar & 100 & - & N & - & - & Employee last name \\ \hline
    position & varchar & 50 & - & N & - & - & Employee position \\ \hline
    street & varchar & 255 & - & N & - & - & Street \\ \hline
    city & varchar & 100 & - & N & - & - & City \\ \hline
    postal\_code & char & 10 & - & N & - & - & Postal code \\ \hline
    country & varchar & 100 & - & N & - & - & Country \\ 
    \hline
\end{tabular}
\label{tab:employee}
\end{table}

\begin{table}[H]
\caption{RoomType Table Attributes}
\begin{tabular}{|l|l|c|c|c|c|c|l|}
    \hline
    \textbf{Name} & \textbf{Data Type} & \textbf{Length} & \textbf{Key} & \textbf{Null} & \textbf{Index} & \textbf{IO} & \textbf{Meaning} \\ \hline
    room\_type\_id & int & - & P & N & A & - & Room type ID \\ \hline
    name & varchar & 100 & - & N & - & - & Room type name \\ \hline
    bed\_count & int & - & - & N & - & - & Bed count \\ 
    \hline
\end{tabular}
\label{tab:roomtype}
\end{table}

\begin{table}[H]
\caption{Room Table Attributes}
\begin{tabular}{|l|l|c|c|c|c|c|l|}
    \hline
    \textbf{Name} & \textbf{Data Type} & \textbf{Length} & \textbf{Key} & \textbf{Null} & \textbf{Index} & \textbf{IO} & \textbf{Meaning} \\ \hline
    room\_id & int & - & P & N & A & - & Room ID \\ \hline
    room\_type\_id & int & - & F(RoomType) & N & - & - & Room type ID \\ \hline
    room\_number & varchar & 20 & - & N & U & - & Room number \\ \hline
    is\_occupied & boolean & - & - & N & - & - & Room occupancy \\ 
    \hline
\end{tabular}
\label{tab:room}
\end{table}

\begin{table}[H]
\caption{Payment Table Attributes}
\begin{tabular}{|l|l|c|c|c|c|c|l|}
    \hline
    \textbf{Name} & \textbf{Data Type} & \textbf{Length} & \textbf{Key} & \textbf{Null} & \textbf{Index} & \textbf{IO} & \textbf{Meaning} \\ \hline
    payment\_id & int & - & P & N & A & - & Payment ID \\ \hline
    total\_accommodation & decimal & - & - & N & - & - & Total accommodation cost \\ \hline
    total\_expenses & decimal & - & - & N & - & - & Total expenses \\ \hline
    payment\_date & date & - & - & A & - & - & Payment date \\ \hline
    is\_paid & boolean & - & - & N & - & - & Payment status \\ 
    \hline
\end{tabular}
\label{tab:payment}
\end{table}

\begin{table}[H]
\caption{Reservation Table Attributes}
\begin{tabular}{|l|l|c|c|c|c|c|l|}
    \hline
    \textbf{Name} & \textbf{Data Type} & \textbf{Length} & \textbf{Key} & \textbf{Null} & \textbf{Index} & \textbf{IO} & \textbf{Meaning} \\ \hline
    reservation\_id & int & - & P & N & A & - & Reservation ID \\ \hline
    guest\_id & int & - & F(Guest) & N & - & - & Guest ID \\ \hline
    room\_id & int & - & F(Room) & N & - & - & Room ID \\ \hline
    employee\_id & int & - & F(Employee) & N & - & - & Employee ID \\ \hline
    creation\_date & date & - & - & N & - & - & Creation date \\ \hline
    check\_in\_date & date & - & - & N & - & - & Check-in date \\ \hline
    check\_out\_date & date & - & - & N & - & - & Check-out date \\ \hline
    payment\_id & int & - & F(Payment) & N & - & - & Payment ID \\ \hline
    status & varchar & 30 & - & N & - & - & Reservation status \\ 
    \hline
\end{tabular}
\label{tab:reservation}
\end{table}

\begin{table}[H]
\caption{Service Table Attributes}
\begin{tabular}{|l|l|c|c|c|c|c|l|}
    \hline
    \textbf{Name} & \textbf{Data Type} & \textbf{Length} & \textbf{Key} & \textbf{Null} & \textbf{Index} & \textbf{IO} & \textbf{Meaning} \\ \hline
    service\_id & int & - & P & N & A & - & Service ID \\ \hline
    name & varchar & 100 & - & N & - & - & Service name \\ \hline
    description & clob & - & - & A & - & - & Service description \\ 
    \hline
\end{tabular}
\label{tab:service}
\end{table}

\begin{table}[H]
\caption{ServiceUsage Table Attributes}
\begin{tabular}{|l|l|c|c|c|c|c|l|}
    \hline
    \textbf{Name} & \textbf{Data Type} & \textbf{Length} & \textbf{Key} & \textbf{Null} & \textbf{Index} & \textbf{IO} & \textbf{Meaning} \\ \hline
    usage\_id & int & - & P & N & A & - & Service usage ID \\ \hline
    reservation\_id & int & - & F(Reservation) & N & - & - & Reservation ID \\ \hline
    service\_id & int & - & F(Service) & N & - & - & Service ID \\ \hline
    quantity & int & - & - & N & - & - & Quantity \\ \hline
    total\_price & decimal & - & - & N & - & - & Total price \\ 
    \hline
\end{tabular}
\label{tab:serviceusage}
\end{table}

\begin{table}[H]
\caption{Feedback Table Attributes}
\begin{tabular}{|l|l|c|c|c|c|c|l|}
    \hline
    \textbf{Name} & \textbf{Data Type} & \textbf{Length} & \textbf{Key} & \textbf{Null} & \textbf{Index} & \textbf{IO} & \textbf{Meaning} \\ \hline
    feedback\_id & int & - & P & N & A & - & Feedback ID \\ \hline
    guest\_id & int & - & F(Guest) & N & - & - & Guest ID \\ \hline
    reservation\_id & int & - & F(Reservation) & N & - & - & Reservation ID \\ \hline
    rating & int & - & - & N & - & - & Rating (1-5) \\ \hline
    comment & clob & - & - & A & - & - & Comment \\ \hline
    feedback\_date & date & - & - & A & - & - & Feedback date \\ 
    \hline
\end{tabular}
\label{tab:feedback}
\end{table}

\begin{table}[H]
\caption{ServicePriceHistory Table Attributes}
\begin{tabular}{|l|l|c|c|c|c|c|l|}
    \hline
    \textbf{Name} & \textbf{Data Type} & \textbf{Length} & \textbf{Key} & \textbf{Null} & \textbf{Index} & \textbf{IO} & \textbf{Meaning} \\ \hline
    sph\_id & int & - & P & N & A & - & Service price history ID \\ \hline
    service\_id & int & - & F(Service) & N & - & - & Service ID \\ \hline
    price & decimal & - & - & N & - & - & Service price \\ \hline
    valid\_from & date & - & - & N & - & - & Valid from \\ \hline
    valid\_to & date & - & - & A & - & - & Valid to \\
    \hline
\end{tabular}
\label{tab:servicepricehistory}
\end{table}

\begin{table}[H]
\caption{RoomTypePriceHistory Table Attributes}
\begin{tabular}{|l|l|c|c|c|c|c|l|}
    \hline
    \textbf{Name} & \textbf{Data Type} & \textbf{Length} & \textbf{Key} & \textbf{Null} & \textbf{Index} & \textbf{IO} & \textbf{Meaning} \\ \hline
    rtph\_id & int & - & P & N & A & - & Room type price history ID \\ \hline
    room\_type\_id & int & - & F(RoomType) & N & - & - & Room type ID \\ \hline
    price\_per\_night & decimal & - & - & N & - & - & Price per night \\ \hline
    valid\_from & date & - & - & N & - & - & Valid from \\ \hline
    valid\_to & date & - & - & A & - & - & Valid to \\
    \hline
\end{tabular}
\label{tab:roomtypepricehistory}
\end{table}

\newpage

\section{DD S05 L01}

\texttt{Task: } Include at least one portable and one non-portable binding in your project

\bigskip

In the project, the following types of bindings are present:

\begin{table}[H]
\caption{Portable and Non-portable Bindings in the Project}
\begin{tabular}{|l|p{6cm}|p{6cm}|}
\hline
\textbf{Table} & \textbf{Portable Binding (Example)} & \textbf{Non-portable Binding (Example)} \\
\hline
Guest & 
Foreign key: \texttt{manager\_id} referencing \texttt{Employee} (supported in all major RDBMS) &
\texttt{NUMBER GENERATED ALWAYS AS IDENTITY} (Oracle-specific auto-increment syntax) \\
\hline
Employee & 
Primary key: \texttt{employee\_id} as PK (portable) &
\texttt{NUMBER GENERATED ALWAYS AS IDENTITY} (Oracle-specific) \\
\hline
RoomType & 
Primary key: \texttt{room\_type\_id} as PK (portable) &
\texttt{NUMBER GENERATED ALWAYS AS IDENTITY} (Oracle-specific) \\
\hline
Room & 
Foreign key: \texttt{room\_type\_id} referencing \texttt{RoomType} (portable) &
\texttt{uq\_room\_number UNIQUE (room\_number)} (constraint name syntax is not portable) \\
\hline
Payment & 
Primary key: \texttt{payment\_id} as PK (portable) &
\texttt{NUMBER GENERATED ALWAYS AS IDENTITY} (Oracle-specific) \\
\hline
Reservation & 
Foreign keys: \texttt{guest\_id}, \texttt{room\_id}, \texttt{employee\_id}, \texttt{payment\_id} (portable) &
\texttt{NUMBER GENERATED ALWAYS AS IDENTITY} (Oracle-specific) \\
\hline
Service & 
Primary key: \texttt{service\_id} as PK (portable) &
\texttt{NUMBER GENERATED ALWAYS AS IDENTITY} (Oracle-specific) \\
\hline
ServiceUsage & 
Foreign keys: \texttt{reservation\_id}, \texttt{service\_id} (portable) &
\texttt{NUMBER GENERATED ALWAYS AS IDENTITY} (Oracle-specific) \\
\hline
Feedback & 
Foreign keys: \texttt{guest\_id}, \texttt{reservation\_id} (portable) &
\texttt{NUMBER GENERATED ALWAYS AS IDENTITY} (Oracle-specific) \\
\hline
ServicePriceHistory & 
Foreign key: \texttt{service\_id} referencing \texttt{Service} (portable) &
\texttt{NUMBER GENERATED ALWAYS AS IDENTITY} (Oracle-specific) \\
\hline
RoomTypePriceHistory & 
Foreign key: \texttt{room\_type\_id} referencing \texttt{RoomType} (portable) &
\texttt{NUMBER GENERATED ALWAYS AS IDENTITY} (Oracle-specific) \\
\hline
\end{tabular}
\end{table}

\newpage

\section{DD S05 L03}

\texttt{Task: } Have at least one M:N relationship without information and one M:N relationship with information in your project

\bigskip

\begin{table}[H]
\caption{Examples of M:N Relationships in the Project}
\begin{tabular}{|l|l|l|p{6cm}|}
\hline
\textbf{Relationship} & \textbf{Join Table} & \textbf{With Information} & \textbf{Description / Additional Attributes} \\
\hline
Guest -- Room & GuestRoom & No & Only tracks which guests have stayed in which rooms; join table contains only \texttt{guest\_id}, \texttt{room\_id} \\
\hline
Reservation -- Service & ServiceUsage & Yes & Tracks which services were used in which reservations; join table contains \texttt{reservation\_id}, \texttt{service\_id}, plus additional attributes like \texttt{quantity}, \texttt{total\_price} \\
\hline
\end{tabular}
\end{table}

\newpage

\begin{landscape}

\section{DD S06 L01}

\texttt{Task: } Incorporate at least one 1:N identifying relationship into your project, with the fact that the transferred foreign key will also be the key in the new table

\bigskip

\begin{table}[H]
\caption{Example of 1:N Identifying Relationship}
\begin{tabular}{|l|l|l|l|}
\hline
\textbf{Parent Table} & \textbf{Child Table} & \textbf{Identifying Foreign Key} & \textbf{Primary Key in Child} \\
\hline
RoomType & RoomTypePriceHistory & room\_type\_id & \begin{tabular}[c]{@{}l@{}}(1) \texttt{rtph\_id} (simple PK)\\ (2) \texttt{room\_type\_id}, \texttt{valid\_from} (composite PK,\\ foreign key is part of PK)\end{tabular} \\
\hline
\end{tabular}
\end{table}

\medskip

In the current design, each \texttt{RoomTypePriceHistory} record has its own primary key (\texttt{rtph\_id}) and a mandatory foreign key (\texttt{room\_type\_id}) referencing \texttt{RoomType}. In an alternative identifying relationship, the primary key of \texttt{RoomTypePriceHistory} could be a composite key (\texttt{room\_type\_id}, \texttt{valid\_from}), making the foreign key also part of the primary key, which is typical for identifying relationships.

\end{landscape}
\newpage

\section{DD S06 L02-04}

\texttt{Task: } Demonstrate that your schema is in first, second, and third normal form

\bigskip

\begin{table}[H]
\caption{Normalization Forms and Tables in the Project}
\begin{tabular}{|l|p{10cm}|}
\hline
\textbf{Normal Form} & \textbf{Tables in the Project} \\
\hline
First Normal Form (1NF) & 
All tables: \texttt{Guest}, \texttt{Employee}, \texttt{RoomType}, \texttt{Room}, \texttt{Payment}, \texttt{Reservation}, \texttt{Service}, \texttt{ServiceUsage}, \texttt{Feedback}, \texttt{ServicePriceHistory}, \texttt{RoomTypePriceHistory}. \newline
\textit{All attributes are atomic, there are no repeating groups or arrays.}
\\
\hline
Second Normal Form (2NF) & 
All tables: \texttt{Guest}, \texttt{Employee}, \texttt{RoomType}, \texttt{Room}, \texttt{Payment}, \texttt{Reservation}, \texttt{Service}, \texttt{ServiceUsage}, \texttt{Feedback}, \texttt{ServicePriceHistory}, \texttt{RoomTypePriceHistory}. \newline
\textit{All non-key attributes are fully functionally dependent on the whole primary key. No partial dependencies exist.}
\\
\hline
Third Normal Form (3NF) & 
All tables: \texttt{Guest}, \texttt{Employee}, \texttt{RoomType}, \texttt{Room}, \texttt{Payment}, \texttt{Reservation}, \texttt{Service}, \texttt{ServiceUsage}, \texttt{Feedback}, \texttt{ServicePriceHistory}, \texttt{RoomTypePriceHistory}. \newline
\textit{No transitive dependencies between non-key attributes. All non-key attributes depend only on the primary key.}
\\
\hline
\end{tabular}
\end{table}

\medskip

All tables in the project schema meet the requirements for 1NF, 2NF, and 3NF.
\newpage

\section{DD S07 L01}

\texttt{Task: } Try to define ARC in your project (can be defined in ORACLE SQL Developer Data Modeler)

\bigskip

An example of an ARC (Alternative Relationship Constraint) in the hotel project could be in the \texttt{Payment} entity. For instance, a payment could be associated either with a reservation or with a service usage, but not both at the same time. This can be modeled by having two nullable foreign keys in the \texttt{Payment} table (\texttt{reservation\_id} and \texttt{usage\_id}) and enforcing that exactly one of them is not null for each payment.

\medskip

\begin{figure}[H]
    \centering
    \includegraphics[width=0.8\textwidth]{ARC.png}
    \caption{Alternative Relationship Constraint}
    \label{fig:er_diagram}
\end{figure}

\newpage

\section{DD S07 L02}

\texttt{Task: } Try to define hierarchical and recursive relations in your project

\bigskip

A hierarchical or recursive relationship can be represented in the \texttt{Employee} table, where an employee can be a manager of other employees. This is modeled by adding a nullable foreign key \texttt{manager\_id} referencing \texttt{employee\_id} in the same table, allowing you to build an organizational hierarchy.

\medskip

\begin{figure}[H]
    \centering

    \begin{minipage}[b]{0.45\textwidth}
        \centering
        \includegraphics[width=\textwidth]{rekurzivni-relation.png}
        \caption{Rekurzivv}
        \label{fig:reservation}
    \end{minipage}
    \hspace{0.05\textwidth}
    \begin{minipage}[b]{0.45\textwidth}
        \centering
        \includegraphics[width=\textwidth]{hiearchic.png}
        \caption{Pokoj}
        \label{fig:room}
    \end{minipage}

\end{figure}

\section{DD S07 L03}

\texttt{Task: } Describe how you record historical data in your system

\bigskip

Historical data in the system is recorded using dedicated history tables, such as \texttt{RoomTypePriceHistory} and \texttt{ServicePriceHistory}. These tables store records of price changes over time, including the start and end dates for each price, allowing the system to track and retrieve historical pricing information for rooms and services.

\newpage

\section{DD S09 L02}

\texttt{Task: } Try journaling in your project, i.e. saving past historical data (for example salary changes, workplace changes, etc.)

\bigskip

Journaling in the hotel project is implemented by using dedicated history tables that store changes over time. For example, the \texttt{RoomTypePriceHistory} and \texttt{ServicePriceHistory} tables record every change in room type prices and service prices, including the period of validity for each price. This allows the system to keep a complete record of all past prices and retrieve historical data as needed.

\newpage

\section{DD S10 L01}

\texttt{Task: } Revise your design according to conventions for the readability of your schema

\bigskip

\section*{Main ERD – Reservation-Centric View}
\begin{figure}[h!]
    \centering
    \includegraphics[width=0.5\textwidth]{reservation.png}
    \caption{Main ER Diagram centered around Reservation}
\end{figure}

\section*{Diagram Description}

This diagram describes the relationships and structure of the hotel reservation system.

\begin{itemize}
  \item \textbf{Guest} stores customer personal data.
  \item \textbf{Employee} represents staff, including those managing reservations.
  \item \textbf{Reservation} links a guest, room, employee, and payment.
  \item \textbf{Room} is typed using \texttt{RoomType}, which in turn has a price history.
  \item \textbf{Service} and \textbf{ServiceUsage} are linked to a reservation.
  \item \textbf{Payment} connects to reservations.
  \item \textbf{Feedback} captures guest opinions tied to reservations.
  \item Tables like \texttt{ServicePriceHistory} and \texttt{RoomTypePriceHistory} track historical pricing.
\end{itemize}


\newpage

\section{DD S10 L02}

\texttt{Task: } Generic modeling – consider, possibly describe or use a generic model of data
structures in your solution, how this approach is more advantageous compared to
traditional data structure design methods

\bigskip

todo

% filepath: c:\Users\ficek\OneDrive\Dokumenty\GitHub\VSB\4_semestr\TDS1\projekt\text.tex
\section{DD S11 L01}

\texttt{Task: } Describe examples of integrity constraints on your project for entities, bindings, attributes, and user-defined integrity

\bigskip

\begin{table}[H]
\caption{Examples of Integrity Constraints in the Project}
\begin{tabular}{|l|p{4cm}|p{8cm}|}
\hline
\textbf{Constraint Type} & \textbf{Example} & \textbf{Description} \\
\hline
Entity Integrity & Primary Key (\texttt{guest\_id}, \texttt{reservation\_id}, etc.) & Ensures each record is uniquely identified and primary keys cannot be NULL or duplicated. \\
\hline
Entity Integrity & IDENTITY Property & \texttt{GENERATED ALWAYS AS IDENTITY} automatically generates unique sequential values for primary keys. \\
\hline
Referential Integrity & Foreign Key (\texttt{fk\_reservation\_guest}, \texttt{fk\_reservation\_room}, \texttt{fk\_serviceusage\_reservation}) & Ensures referenced records exist and prevents orphaned records in related tables. \\
\hline
Domain Integrity & NOT NULL (\texttt{firstname}, \texttt{lastname}, \texttt{email}) & Ensures essential attributes must always have a value. \\
\hline
Domain Integrity & Data Type (\texttt{VARCHAR2(100)}, \texttt{NUMBER(10,2)}, \texttt{DATE}) & Restricts attribute values to specific types and formats. \\
\hline
Domain Integrity & DEFAULT Value (\texttt{registration\_date}, \texttt{creation\_date}) & Automatically assigns a value (e.g., \texttt{SYSDATE}) if none is provided. \\
\hline
User-Defined Integrity & UNIQUE (\texttt{uq\_room\_number}) & Ensures each room has a unique room number. \\
\hline
User-Defined Integrity & CHECK (e.g., \texttt{rating} 1--5, \texttt{is\_occupied} 0/1, \texttt{is\_paid} 0/1) & Enforces business rules and valid value ranges for attributes. \\
\hline
User-Defined Integrity & Custom Business Rules & E.g., check-out date must be after check-in date (enforced by triggers or application logic). \\
\hline
\end{tabular}
\end{table}

\medskip

These integrity constraints work together to ensure data consistency, prevent invalid data entry, and maintain the overall reliability of the hotel management database system.


\newpage

\section{DD S11 L02-04}

\texttt{Task: } Generate a relational schema from your conceptual model and note the changes that
have occurred in the schema and why

\begin{figure}[H]
    \centering
    \includegraphics[width=0.8\textwidth]{Relational_1.png}
    \caption{ER Diagram for Hotel Database}
    \label{fig:er_diagram}
\end{figure}

\end{document}